\section{Features}

This usecase intends to demonstrate how patterns of
computation that consist of a loop with sequential
dependency, i.e.~each iteration is dependent on the
previous iteration, can be expressed as a set of 
concurrent computations. The interaction between 
parallel components can be expressed using session types.


To clarify the above motivation, consider the following general pattern
sequential code:
%
\begin{lstlisting}
  int x = init();
  for(int i = 0; i < N; i++) {
    x = compute(x);
    resourceConsumingFun(x);
  }
\end{lstlisting}
%
Each iteration of the above loop is dependent on
the previous iteration to produce data \lstinline|x|.
It then uses the produced data \lstinline|x| to perform
a resource consuming computation which is assumed to
be independent between iterations.

Although the above algorithm is inherently sequential,
it can be expressed as concurrent computation between
two computation nodes. The code for the first node
(Node A) is:
%
\begin{lstlisting}[caption={Code for Node A}]
  int x = init();
  for(int i = 0; i < N; i++) {
    x = compute(x);
    b.send(NEXT);
    b.send(x);
    resourceConsumingFun(x);
    x = b.recv();
  }
  b.send(END);
\end{lstlisting}
%
The code for the second node (Node B) is:
%
\begin{lstlisting}[caption={Code for Node B}]
  loop: do{
    switch(a.recv()) {
      case NEXT:
        x = a.recv();
        x = compute(x);
        resourceConsumingFun(x);
        a.send(x);
        continue loop;
      case END:
        break loop;
    }
  } while(true);
\end{lstlisting}

The above code splits the iterations of the loop across
nodes A and B. The fact that each iteration is doing
an independent time consuming operation may lead to
an optimised concurrent implementation of the serial
code due to the fact that different \lstinline|resourceConsumingFun(x)|
function calls will be happening in parallel.

The communication that takes place in the above scenario
can be described using a multiparty session interaction
between Nodes A and B:
\begin{lstlisting}[caption={Global Protocol}]
global protocol Conc(role A, role B) {
  rec X {
    choice at A {
      NEXT(long) from A to B;
      NEXT(long) from B to A;
      continue X;
    } or {
      END() from A to B;
    }
  }
}
\end{lstlisting}
%
The motivation for developing such an algorithm is:
i) exploit any parallelism that may exist between
sequentially dependent iterations of an algorithm; and
ii) use session types to show that the interaction
patterns of such transformations can be expressed in
a structure and descriptive way, that follows all the
properties proposed by session types. 

An instance of the above problem, without the time consuming
computation, is the classic Fibonacci algorithm, where each
iteration for computing a Fibonacci number is dependent
on the previous two iterations of the algorithm.



\begin{comment}
Consider the following general pattern of serial code:
%
\begin{lstlisting}
  data x = init();
  for(int i = 0; i < n; i++) {
    x = f(x);
    process(x);
  }
\end{lstlisting}
%
The above code describes the typical case where a loop
computes a value \lstinline|x| based on the value
\lstinline|x| computed in the previous iteration.
It further calls a general method \lstinline|process(data x)|
to process the computed value.

The serial Fibonacci algorithm is an instance of the above
pattern.
%
\begin{lstlisting}
  int x = 0;
  int y = 1;
  for(int i = 0; i < n; i++) {
    int z = x;
    y += x;
    x = z;
  }
\end{lstlisting}
%
By observing the above serial Fibonacci algorithm and
the concurrent Fibonacci algorithm and by assuming
that \lstinline|process(data x)| is not dependent on shared resources,
we can generalise a two node concurrent implementation of the
former general programming pattern as:
%
\begin{lstlisting}[caption={Code for Node A}]
  data x = init();
  for(int i = 0; i < n; i++) {
    b.send(Choice.NEXT);
    b.send(x);
    process(x);
    x = a.recv();
    if(i++ < n)
      x = f(x);
  }
    b.send(Choice.END);
\end{lstlisting}
%
and
%
%
\begin{lstlisting}[caption={Code for Node B}]
  data x = init();
  loop: do {
    switch(a.recv()) {
      case Choice.NEXT:
        x = a.recv();
        x = foo(x);
        a.send(x);
        process(x);
        continue loop;
      case Choice.END:
        break loop;
    }
  } while(true);
\end{lstlisting}

The global type of the above protocol is the same as
the protocol for the concurrent Fibonacci algorithm,
up-to the type of the carried data.

The generalisation gives evidence that session types
can be used to describe the parallelisation structure
of patterns of serial code. This structure is expressed
by means of communication interaction.

Someone might argue that, because of the serial dependency
nature of the serial algorithm there is a reduced speed-up
in the parallelised code, due to the fact that we are using
more resources. However, if method \lstinline|process(data x)|
is resource demanding enough we can manage to obtain the analogous
speed-up.
\end{comment}


\begin{comment}
In this section we develop an informal correspondence
between the concurrent Fibonacci code and other
sequential implementations of the Fibonacci algorithm.
Considered the following recursive code:
%
\begin{lstlisting}
  int fibonacci(int n, int x, int y) {
    if(n <= 1)
      return x;
    return fibonacci(n - 1, x + y, x);
  }

  int fibonacci(int n) {
    return fibonacci(n, 1, 0)
  }
\end{lstlisting}
%
The call for computing the kth Fibonacci number
would be \lstinline|fibonacci(k);|.
We can understand the concurrent implementation,
if we consider communications taking place instead
of recursive calls and returns.
%This procedure
%gives rise to the 
The concurrent code has also the ability
to maintain a state closure that will be used
on a later interaction.

A serial procedure for computing Fibonacci
is implemented by the following code:
%
\begin{lstlisting}
  int x = 0;
  int y = 1;
  for(int i = 0; i < n; i++) {
    int z = x;
    y += x;
    x = z;
  }
\end{lstlisting}
%
Each iteration of the above code corresponds to
a recursive Fibonacci call.
The above code can be generalised to a pattern
of the form
\begin{lstlisting}
  for(int i; i < n; i++) {
    x = f(x);
  }
\end{lstlisting}
\end{comment}
