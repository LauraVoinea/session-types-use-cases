\section{Typestate and \Mungo implementation}
In this section we present the typestate specification
and we describe the implementation of the Fibonacci
algorithm in the \Mungo framework.

\subsection{Typestate for \Mungo}
\label{sec:typestate}
The concurrent Fibonacci algorithm is implemented in \Mungo.
For the purpose of this implemention,
we translate the Scribble local protocols for roles
\A and \B to \Mungo typestate syntax.
The translation can be done automatically;
an automatic translation first derives an
enumeration type:

\begin{lstlisting}[caption={Enumeration for the communication choice}]
enum Choice implements java.io.Serializable {
  FIBONACCI, STOP;
}
\end{lstlisting}

The enumeration represents the
labels used by the Scribble protocol
to make a choice at a role. In this
case we have labels \lstinline|FIBONACCI| and
\lstinline|STOP| that correspond to the choices
made by role \A.

The local projection of role \A is automatically translated to typestate:

\begin{lstlisting}[caption={Typestate for Role \A}]
typestate AFibonacci {
  State0 = {
    void send_fibonacciToB() : State1,
    void send_stopToB() : end,
   }

  State1 = {
    void send_longToB(long): State2
  }

  State2 = {
    int receive_longFromB(): State0
  }
}
\end{lstlisting}

The initial state is \lstinline|State0| where
role \A chooses whether to invoke method
\lstinline|void send_fibonacciToB()| that
will send enumeration label \lstinline|FIBONACCI| to role \B
and then proceed to \lstinline|State1|. The other
choice for role \A is to invoke method
\lstinline|void send_stopToB()| that sends
enumeration label \lstinline|END| to role \B
and then ends the protocol.
\lstinline|State1| requires that role \A invokes
method \lstinline|void send_longToB(long)| that
is used to send a \lstinline|long| integer to role
\B and then proceed to \lstinline|State2|. In 
\lstinline|State2| role \A should invoke method
\lstinline|long receive_longFromB()| that will
receive a \lstinline|long| integer from role \B.
The protocol then continues in \lstinline|State0|.


The local projection of role \B is automatically translated to typestate:

\begin{lstlisting}[caption={Typestate for Role \B}]
typestate BFibonacci {
  State0 = {
    void receive_ChoiceFromA() : <FIBONACCI: State1, STOP: end>
  }

  State1 = {
    int receive_intFromA(): State2
  }

  State2 = {
    void send_intToA(int): State0
  }
}
\end{lstlisting}

The typestate for role \B starts with \lstinline|State0|
that performs an external choice on method
\lstinline|void receive_ChoiceFromA()| that will receive
one of the enumeration labels of the \lstinline|Choice|
if the label received is \lstinline|FIBONACCI| the state
will proceed to \lstinline|State1| or if the label
received is \lstinline|STOP| the protocol will terminate.
States \lstinline|State1| and \lstinline|State2| perform
a dual communication pattern than the respective states
in the typestate of role \A.


\subsection{\Mungo Implementation}

Each of the two typestates, \lstinline|AFibonacci| and
\lstinline|BFibonacci| are implemented by classes
\lstinline|ARole| and \lstinline|BRole| respectively.
The constructors of the two classes synchronise
on an object with a socket interface (class \lstinline|ARole|
is the server side) and initialise their private socket
field.
The two classes implement
the methods defined their respective protocols. The
method implementations perform send and receive actions
over their socket field. The two classes are then
used by classes \lstinline|A| and \lstinline|B|, respectively.
Classes \lstinline|A| and \lstinline|B| implement the \lstinline|Runnable|
interface so they can be instantiated inside a thread object.
Each class creates a new \lstinline|ARole| (\lstinline|BRole|, respectively)
object and it is forced to use it as specified by its typestate
protocol.
Two objects of the classes \lstinline|A| and \lstinline|B| are
then run into separate threads that coordinate and execute the
Fibonacci algorithm.

\begin{comment}
We could refine the automated code to be more readable:
\begin{lstlisting}
typestate AFibonacci {
  Fibonacci = {
    void send_fibonacciToB() : PingPong,
    void send_stopToB() : end,
  }

  PingPong	=	{
    void send_intToB(int): { 
      int receive_intFromB(): Fibonacci
    }
  }
}
\end{lstlisting}

\begin{lstlisting}
typestate BFibonacci {
  Fibonacci = {
    void receive_ChoiceFromA() : <FIBONACCI: PingPong, STOP: end>
  }

  PingPong = {
    int receive_intFromA(): {
      void send_intToA(int): State0
    }
  }
}
\end{lstlisting}
\end{comment}

