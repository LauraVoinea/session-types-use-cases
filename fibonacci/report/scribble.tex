\section{Scribble Protocol}

\paragraph{Global Protocol}
We give the Scribble protocol, that formally specifies the
global communication described in \secref{subsec:algorithm}.
%
\begin{lstlisting}[caption={Global Protocol}]
global protocol Fibonacci(role A, role B) {
  rec Fib {
    choice at A {
      fibonacci(long) from A to B;
      fibonacci(long) from B to A;
      continue Fib;
    } or {
      stop() from A to B;
    }
  }
}
\end{lstlisting}

The above protocol starts by declaring a recursive block
with recursive variable \lstinline|Fib|.
Inside the recursion role \A decides whether to send
a \lstinline|long| integer (Java primitive data-type)
to role \B or send an \lstinline|stop| to role \B.
In the former case role's \A message will be followed
by role \B sending a \lstinline|long| integer. The protocol
then performs a recursion on the recursive variable.
In role's \A latter choice the protocol terminates.

\paragraph{Local projection}
The global protocol is used to project the communication
behaviour of roles \A and \B.

\begin{lstlisting}[caption={Local Protocol for Role \A}]
local protocol Fibonacci at A(role A, role B) {
  rec Fib {
    choice at A {
      fibonacci(long) to B;
      fibonacci(long) from B;
      continue Fib;
    } or {
      stop() to B;
    }
  }
}
\end{lstlisting}

The code above describes the global protocol from role's \A
point of view. It give the exact communication actions that
role \A needs to do in order to respect the global protocol.

\begin{lstlisting}[caption={Local Protocol for Role \B}]
local protocol Fibonacci at B(role A, role B) {
  rec Fib {
    choice at A {
      fibonacci(long) from A;
      fibonacci(long) to A;
      continue Fib;
    } or {
      stop() from A;
    }
  }
}
\end{lstlisting}

Dually, the code above describes the global protocol from
the point of view of role \B. Role \B needs to perform
the dual communication actions with respect to role \A
in order to respect the global protocol.



