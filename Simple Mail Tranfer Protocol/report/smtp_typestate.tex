% !TEX root = smtp_main.tex
\section{Implementation with \StMungo and \Mungo}
\label{sec: smtp_implementation}
In this section we describe the implementation
details of the \lstinline|SMTP_C| protocol.

%
For the purpose of implementing the SMTP
algorithm in \Mungo,
we first translate the Scribble local protocol for the client, namely the \lstinline|SMTP_C| protocol, into \Mungo typestate syntax.

The translation is be done automatically by using the \StMungo tool,
which produces in addition to the typestate,
enumeration types for every choice in the \lstinline|SMTP_C| protocol as well as a skelatal API.

Running the StMungo tool on \lstinline|SMTP_C| produces the files
\lstinline|CProtocol.mungo|,
\lstinline|CRole.mungo| and \lstinline|CMain.mungo|. The first of these, \lstinline|CProtocol|, captures the local protocol as a typestate definition.
\begin{lstlisting}[numbers=left]
typestate CProtocol {
  State0 = {
    String receive_220StringFromS(): State1 }
  State1= {
    void send_EHLOToS(): State2,
    void send_QUITToS(): State39 }
  State2 = {
    void send_ehloStringToS(String): State3 }
  ...
  State29 = {
    void send_SUBJECTToS() : State30,
    void send_DATALINEToS(): State31,
    void send_ATADToS() : State32 }
  ...
  State39 = {
    void send_quitStringToS(String): end }
}
\end{lstlisting}
%%
The internal choice made at role \lstinline|C| (lines 11-20 of
\lstinline|SMTP_C|), is translated into a set of send methods,
one for each branch of the choice (lines 11-13 in
\lstinline|CProtocol|).


In this implementation, since communication occurs on Java sockets,
we declare and create a new socket to connect to the \lstinline|gmail| server.
This is given in lines 2 and 5 in \lstinline|CRole|, respectively.
\begin{lstlisting}[numbers=left]
public class CRole typestate CProtocol {
  private Socket socketS = null;
  ...
  public CRole() {
    socketS = new Socket("smtp.gmail.com", 587);
    ...
  }
   /* CProtocol  method definitions */
}
\end{lstlisting}

There is a close correspondence between the text-based commands in SMTP
and the method calls in \Mungo, described as follows.
Consider the command
``\lstinline|SUBJECT: Hello World|'', which is an atomic command
starting with the keyword \lstinline|SUBJECT| and followed by the
subject line, in this case ``Hello World''.

%
\StMungo and \Mungo framework use an intermediate layer to split the above command into
two separate method calls, as shown in lines 8-10 in \lstinline|CMain|.
The first method being \lstinline|send_SUBJECTToS()|, selects the command \lstinline|SUBJECT|.
The second method being
\lstinline|send|$\_$\lstinline|subjectStringToS(``Hello World'')|, completes and sends the message
``\lstinline|SUBJECT: Hello World|''.
The intermediate layer is also used when receiving a command from the server,
by splitting it into a choice and the corresponding text.

Finally, \lstinline|CMain.mungo| contains the \lstinline|main| method
where the \lstinline|CRole| object is created and used to implement the
client logic.
\begin{lstlisting}[numbers=left]
public static void main(String[] args) {
  CRole currentC =  new CRole();
  ...
  _Z3: do{
    ...
    switch(/*label to be sent*/) {
      case /*SUBJECT*/:
        currentC.send_SUBJECTToS();
        String subject = // input subject;
        currentC.send_subjectStringToS(/*subject*/);
        continue _Z3;
      case /*DATALINE*/
      case /*ATAD*/:
        currentC.send_ATADToS();
        currentC.send_atadStringToS(/*single dot*/);
        String _250msg = currentC.receive_250StringFromS();
        continue _Z1;
    }
  }
  while(true);
}
\end{lstlisting}
%%
The \lstinline|CMain| file should be thought of as a skeleton or a
template for the user, who is then able to customise it, for example by
adding SSL connection code for authentication with the \lstinline|gmail|
server.