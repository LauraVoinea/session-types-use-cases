% !TEX root = smtp_main.tex
%%
\section{Scribble Protocol}
\label{sec: smtp_scribble}

We use Scribble to define the global
protocol for the SMTP protocol, based
on Hu's work~\cite{HuR:smtp}.

%%
\subsection{Global Protocol}
\label{sec: smtp_global}

%
\begin{lstlisting}[numbers=left]
global protocol SMTP(role S, role C) {
 // Global interaction between server and client.
  _220(String) from S to C;
   choice at C {
    ehlo(String) from C to S;
    rec X {
      choice at S {
        _250dash(String) from S to C;
         continue X;
      } or {
       _250(String) from S to C;
        choice at C {
         ...
         rec X1 {
           ...
           choice at S {
             _250dash(String) from S to C;
             continue X1;
          } or {
            250(String) from S to C;
            ...
            rec Z1 {
              ...
              data(String) to S;
              ...
              rec Z3 {
                choice at C {
                  subject(String) from C to S;
                  continue Z3;
                } or {
                  dataline(String) form C to S;
                  continue Z3;
                } or {
                  atad(String) from C to S;
                  _250(String) from S to C;
                  continue Z1; }
                }
              }   ...  
           }  
           ... 
        }
      }
    }
  }
 } or  { quit(String) from C to S; }
}
\end{lstlisting}


The above Scribble protocol formally specifies
the description in \secref{sec: smtp_description}.
The global protocol \lstinline|SMTP| specifies
an interaction between two roles being the client ant the server of the protocol.
The protocol is then projected to the
individual communication specifications of each 
of the participant roles.


%%
\subsection{Local projection}
\label{subsec: smtp_local}

The global protocol is used to project the communication
behaviour of roles \lstinline|C| and \lstinline|S|.
We focus on the projection of the client as we are interested in interacting with a real server.

This part of SMTP describes a loop (\lstinline|rec Z1|) where the client chooses
among the messages \lstinline|SUBJECT|, to send the subject,
\lstinline|DATALINE|, to send a line of text, or \lstinline|ATAD| to terminate the e-mail by sending a dot.

\begin{lstlisting}[numbers=left]
local protocol SMTP_C(role S, self C) {
  _220(String) from S;
  choice at C{
    ehlo(String) to S;
    ...
    rec Z1 {
      ...
      data(String) to S;
      ...
      rec Z3 {
        choice at C {
          subject(String) to S;
          continue Z3;
        } or {
          dataline(String) to S;
          continue Z3;
        } or {
          atad(String) to S;
          _250(String) from S;
          continue Z1; }}}
      ...
  } or { quit(String) to S; }
}
\end{lstlisting}
%
