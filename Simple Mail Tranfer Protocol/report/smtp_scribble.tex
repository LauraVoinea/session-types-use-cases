% !TEX root = main.tex
\section{Scribble Protocol}

We use Scribble to define the global
protocol for the SMTP protocol, based
on Hu's work~\cite{HuR:smtp}.

\subsection{Global Protocol}

%
\begin{lstlisting}[numbers=left]
global protocol SMTP(role S, role C) {
 // Global interaction between server and client.
 _220(String) from S to C;
 choice at C {
   ehlo(String) from C to S;
   rec X {
     choice at S { _250dash(String) from S to C; continue X; }
     or {
       _250(String) from S to C;
       choice at C { ...
         rec X1 { ...
           choice at S { _250dash(String) from S to C; continue X1; }
           or { 250(String) from S to C; ... 
             rec Z1 { ... data(String) to S; ...
               rec Z3 {
                 choice at C { subject(String) from C to S; continue Z3; }
                 or { dataline(String) form C to S;  continue Z3; }
                 or { atad(String) from C to S; _250(String) from S to C; 
                     continue Z1; } } }   ...
        }   ...  } } } }
 } or  { quit(String) from C to S; }
}
\end{lstlisting}


The above Scribble protocol formally specifies
the description in \secref{sec: smtp_description}.
The global protocol \lstinline|SMTP| specifies
an interaction between two roles being the client ant the server of the protocol.
The protocol is then projected to the
individual communication specifications of each 
of the participant roles.



\subsection{Local projection}
The global protocol is used to project the communication
behaviour of roles $A$ and B.


\begin{lstlisting}[caption={Local Protocol for Role \BuyerOne}]
local protocol Bookstore at Buyer1(role Buyer1,
                                   role Buyer2,
                                   role Seller) {
  book(title) to Seller;
  book(price) from Seller;
  quote(contribution) to Buyer2;

  choice at Buyer2 {
    agree() fom Buyer2;
    transfer(money) to Seller;
  } or {
    quit() from Buyer2;
  }
}
\end{lstlisting}

The above code describes the global protocol from the
point of view of \BuyerOne. It specifies the
communication actions that \BuyerOne needs to implement
in order to respect its part of the global protocol. 

\begin{lstlisting}[caption={Local Protocol for Role \BuyerTwo}]
local protocol Bookstore at Buyer2(role Buyer1,
                                   role Buyer2,
                                   role Seller) {
  quote(contribution) from Buyer1;

  choice at Buyer2 {
    agree() to Buyer1, Seller;
    transfer(money) to Seller;
  } or {
    quit() to Buyer1, Seller;
  }
}
\end{lstlisting}

Similarly, the code above describes the global protocol
from the point of view of role \BuyerTwo. Note that any
interaction of the global protocol that does not involve
\BuyerTwo is ommited from the local protocol of \BuyerTwo.

\begin{lstlisting}[caption={Local Protocol for Role \Seller}]
local protocol Bookstore at Seller(role Buyer1,
                                   role Buyer2,
                                   role Seller) {
  book(title) from Buyer1;
  book(price) to Buyer1;

  choice at Buyer2 {
    agree() fom Buyer2;
    transfer(money) from Buyer1;
    transfer(money) from Buyer2;
  } or {
    quit() from Buyer2;
  }
}
\end{lstlisting}

Finally, the \Seller local protocol is projected as above.
