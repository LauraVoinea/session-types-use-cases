\section{Implementation}
In this section we describe the implementation
details of the Bookstore scenario.

\subsection{\Mungo Implementation}

For the purpose of implementing the Bookstore
algorithm in \Mungo,
we translate the Scribble local protocols for roles
$A$ and $B$ to \Mungo typestate syntax.
The translation can be done automatically;
an automatic translation first derives an
enumeration type:

\begin{lstlisting}[caption={Enumeration for the communication choice}]
enum Choice implements java.io.Serializable {
  AGREE, QUIT;
}
\end{lstlisting}

The enumeration represents the
labels used by the Scribble protocol
to perform choice between roles. In this
case we have labels \lstinline|AGREE| and
\lstinline|QUIT| that correspond to the choices
made by role \BuyerTwo.

The local projection of role \BuyerOne is automatically translated to typestate:

\begin{lstlisting}[caption={Typestate for Role \BuyerOne}]
typestate Buyer1Protocol {
  State0 = {
    void send_bookToSeller(Title): State1
  }

  State1 = {
    Price receive_bookFromSeller(): State2
  }

  State2 = {
    void send_quoteToBuyer2(Contribution): State3
  }

  State3 = {
    Choice receive_ChoiceFromBuyer2(): <AGREE: State4, QUIT: end>
  }

  State4 = {
    void send_transferToSeller(Money): end
  }
}
\end{lstlisting}

The initial state is \lstinline|State0| where
role \BuyerOne uses method \lstinline|void send_bookToSeller(title)|
to send a book title to \Seller
and proceed to \lstinline|State1|. In \lstinline|State1|
\BuyerOne receives a price from the \Seller for the book via
the invocation of method \lstinline|Price receive_bookFromSeller()|
and proceeds to \lstinline|State2|, where
it calls method \lstinline|void send_quoteToBuyer2(Contribution)|
to inform \BuyerTwo about a  possible contribution. The latter
action results in \lstinline|State3|.
In \lstinline|State3| \BuyerOne calls method
\lstinline|Choice receive_ChoiceFromBuyer2()| and awaits
for a choice from \BuyerTwo; the choice \lstinline|AGREE|
proceeds the protocol to \lstinline|State4| and
choice \lstinline|QUIT| ends the protocol. Finally,
in \lstinline|State4| \BuyerOne uses method
\lstinline|void send_transferToSeller(Money)| to
send transfer money to the \Seller.


The local projection of role \BuyerTwo is automatically translated to typestate:

\begin{lstlisting}[caption={Typestate for Role \BuyerTwo}]
typestate BuyerTwoProtocol{
  State0 = {
    Contribution receive_quoteFromBuyer1(): State1

  State1 = {
    void send_AGREEToBuyer1: State2,
    void send_QUITToBuyer1: State4
  }

  State2 = {
    void send_AGREEToSeller: State3
  }

  State3 = {
    void send_transferToSeller(Money): end
  }

  State4 = {
    void send_QUITToSeller(): end
  }
}
\end{lstlisting}


\begin{lstlisting}[caption={Typestate for Role \Seller}]
  State0 = {
    Title receive_bookFromBuyer1(): State1
  }

  State1 = {
    void send_bookToBuyer1(Price): State2
  }

  State2 = {
    Choice receive_ChoiceFromBuyer2(): <AGREE: State3, QUIT: end>
  }

  State3 = {
    Money receive_transferFromBuyer1(): State4
  }

  State4 = {
    Money receive_transferFromBuyer2(): end
  }
\end{lstlisting}

\subsubsection{Implementation in \Mungo}

We have implemented two versions of the Bookstore scenario in
\Mungo, based on the underline communication layer. The first
version implements a socket interface on top of a shared
memory structure. The second versions uses the Java socket
interface from the \code{Java.net.*} package.

The shared memory implementation is used for demonstration purposes
does not implement concurrent components; roles \BuyerOne,
\BuyerTwo and \Seller are instantiated as objects in a single
method in the \code{Main.java} file and they are interacting
in a sequential discipline.

The socket implementation is a concurrent implementation and
requires running three different JVM processes to instatiate
the roles \BuyerOne, \BuyerTwo, and \Seller. The \Seller role
acts as the server side for both the sockets of
\BuyerOne, and \BuyerTwo. Similarly, \BuyerTwo acts as the server
for the socket connection between \BuyerOne and \BuyerTwo.
Classes \code{Buyer1Main.java}, \code{Buyer2Main.java}, and
\code{SellerMain.java} implement the communication interaction
of the three roles, respectively, as it is described by the
typestate protocol of each role.







